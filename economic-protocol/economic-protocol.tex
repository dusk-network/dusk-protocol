\documentclass[twocolumn, nofootinbib]{revtex4-2}

\usepackage{array}

\begin{document}
    \title{Dusk Economic Model}
    \author{Emanuele Francioni}
    \email{emanuele@dusk.network}
    \author{Matteo Ferretti}
    \email{matteo@dusk.network}
    \affiliation{Dusk Network}
%    \date{\today}

    \begin{abstract}
        The Economic Model encompasses the mechanisms enabling Smart Contract
        owners to create economic value through the services they offer.
        It consists of levying service fees, offsetting gas costs for users,
        and optimizing gas payments for improved profitability. Essentially, the
        Economic Model allows service providers to be productized and profit.
    \end{abstract}

    \maketitle

    \section{Terminology}\label{sec:terminology}
    \begin{itemize}
        \item \textbf{contract}: The smart contract method activated by the user
              through a transaction.
        \item \textbf{user}: The party initiating the \textbf{contract} transaction.
        \item \textbf{gas}: A unit measure of computational resource.
              In this scope, \textbf{gas limit} indicates the maximum amount of
              resources allocated by a user or a contract to perform a
              computation, \textbf{gas consumed} measures incurred costs of
              computation executed, \textbf{gas unspent} is the difference
              between \textbf{gas limit} and \textbf{gas consumed}.
              If a computation requires more resources than those allocated,
              then we incur in an \textbf{insufficient gas} error.
        \item \textbf{gas price}: This is the amount of DUSK that a sender is
              willing to spend per unit of gas (unit of gas are specified in
              Lux, where 1 Lux equals 10\textsuperscript{-9} DUSK). The higher
              the \textbf{gas price}, the more incentive Block Generators have
              in including the transaction in the next block, so transactions
              with high gas prices are generally confirmed more quickly.
              Because of this, \textbf{gas price} generally determines the
              transaction priority.
        \item \textbf{ICC\footnote{ICCs cannot be called directly. As such, the
              contract methods should be tagged as C2C (contract-to-contract) or
              direct. The former can only be called only by another contract,
              while the latter can only be called by a user (through a
              transaction).}} (Inter-Contract Call): Interaction between
              multiple contracts.
              It is also used interchangeably for those contract methods that
              can only be called by other contract.
        \item \textbf{fee}: The fee applied by the smart contract to the user, when the
              former requires payment for its services.
              The \textbf{fee} is known by the Transfer Contract (changeable
              through a transaction), but needs to be communicated to the user,
              who approves/signes it.
              This way the possibility of fee malleability is removed, thus
              preventing a bait-and-switch attack where a contract could change
              the fee with a high priority transaction and drain the user's
              funds.
        \item \textbf{contract account}: In the scenario where the contract pays
              for gas on behalf of the user, the \textbf{contract account} is
              where fees are accrued and gas is paid
        \item \textbf{provider}: An intermediary service entity between a
              \textbf{user} and a \textbf{contract}.
    \end{itemize}

    \section{Specifications}\label{sec:specifications}
    \begin{table*}[t]
    \begin{tabular}{|m{0.15\linewidth}|m{0.1\linewidth}|m{0.45\linewidth}|m{0.3\linewidth}|}
        \hline
        \textbf{Scenario} & \textbf{Contract Fee} &
        \textbf{Normal Flow} & \textbf{Insufficient Gas} \\
        \hline
        \hline
        User pays gas & No &
        \begin{itemize}
            \item 70\% \textbf{gas unspent} returned to user
            \item 30\% \textbf{gas unspent} paid to executed contracts depending
                  on the \textbf{gas limit} allocated across ICC
            \item 100\% \textbf{gas consumed} paid to the Block Generator
        \end{itemize} &
        \begin{itemize}
            \item 100\% \textbf{gas consumed} paid to the Block Generator
        \end{itemize} \\
        \hline
        User pays gas; contract applies fee & Yes &
        \begin{itemize}
            \item Like Scenario 1
            \item \textbf{fee} paid to the charging contract
        \end{itemize} &
        \begin{itemize}
            \item 100\% \textbf{gas consumed} paid to the Block Generator
            \item \textbf{fee} returned to \textbf{user}
        \end{itemize} \\
        \hline
        Contract pays gas; contract applies fee & Yes &
        \begin{itemize}
            \item \textbf{gas consumed} charged entirely to \textbf{contract}
            \item 30\% \textbf{gas unspent} paid to executed contracts depending
                  on the \textbf{gas limit} allocated across ICC
            \item 70\% \textbf{gas unspent} returned to contract
            \item 100\% \textbf{gas consumed} paid to the Block Generator.
            \item \textbf{fee} paid to the charging contract
        \end{itemize} &
        \begin{itemize}
            \item 100\% \textbf{gas consumed} locked for a number of epochs
            \item \textbf{fee} gets refunded to the \textbf{user}
        \end{itemize} \\
        \hline
        Obfuscated transactions & Variable, \% of an obfuscated amount & As in
        Scenario 3, \textbf{fee} calculated as \% of transacted amount &
        Additional considerations needed \\
        \hline
        Autocontracts & No & TBD & TBD \\
        \hline
    \end{tabular}
    \end{table*}

    \subsection{Scenario 1: User pays gas}\label{subsec:scenario-1}
    \subsection{Scenario 2: User pays gas, contract applies fee}\label{subsec:scenario-2}
    \subsection{Scenario 3: Contract pays gas, contract applies fee}\label{subsec:scenario-3}
    \subsection{Scenario 4: Contract charges a percentage of an obfuscated amount}\label{subsec:scenario-4}
    \subsection{Scenario 5: Autocontracts}\label{subsec:scenario-5}
\end{document}
