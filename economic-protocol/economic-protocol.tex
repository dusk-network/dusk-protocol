\documentclass[twocolumn, nofootinbib]{revtex4-2} %uncomment to hide comments
%\documentclass[draft, twocolumn, nofootinbib]{revtex4-2} %uncomment to see comments

% General comments:
% contract vs. function in a contract
% crossover
% direct

%\usepackage[pagebackref=true]{hyperref}

\usepackage{multirow}
\usepackage{array}
\usepackage[colorinlistoftodos, textwidth = 20mm, obeyDraft]{todonotes}
\newcommand{\mbi}[1]{{\todo[inline, size=\small, color=olive!40]{\textbf{Marta}: #1}}}

\usepackage{xspace}
\newcommand{\dusk}{{\footnotesize\textsf{DUSK}}\xspace}

\usepackage{setspace}

% Emphasized words
\newcommand{\emphasize}[1]{\textbf{#1}\xspace}
\newcommand{\blockgenerator}{\emphasize{block generator}}
\newcommand{\contract}{\emphasize{contract}}
\newcommand{\contractaccount}{\emphasize{contract account}}
\newcommand{\ctoc}{\emphasize{C2C}}
\newcommand{\direct}{\emphasize{direct}}
\newcommand{\fee}{\emphasize{fee}}
\newcommand{\gas}{\emphasize{gas}}
\newcommand{\gasprice}{\emphasize{gas price}}
\newcommand{\gaslimit}{\emphasize{gas limit}}
\newcommand{\gasconsumed}{\emphasize{gas consumed}}
\newcommand{\gasunspent}{\emphasize{gas unspent}}
\newcommand{\insufficientgas}{\emphasize{insufficient gas}}
\newcommand{\Insufficientgas}{\emphasize{Insufficient gas}}
\newcommand{\icc}{\emphasize{ICC}}
\newcommand{\provider}{\emphasize{provider}}
\newcommand{\transfercontract}{\emphasize{transfer contract}}
\newcommand{\user}{\emphasize{user}}
\newcommand{\users}{\emphasize{users}}
\newcommand{\emphp}{\emphasize{p}}

\begin{document}
    \title{Economic Protocol In Dusk Smart Contracts}
    \author{Emanuele Francioni}
    \email{emanuele@dusk.network}
    \author{Matteo Ferretti}
    \email{matteo@dusk.network}
    \affiliation{Dusk Network}
%    \date{\today}

%    \begin{abstract}
%    	Abstract.
%    \end{abstract}

    \maketitle

%    \tableofcontents

	\section*{Introduction}\label{sec:introduction}
	\mbi{I moved the abstract to this initial section.}
	The economic model of Dusk encompasses the mechanisms enabling smart
	contract owners to create economic value through the services they offer.
	It consists of levying service fees, offsetting gas costs for users,
	and optimizing gas payments for improved profitability.
	Essentially, the economic model allows service providers to be
	productized and for profit.

    \section*{Terminology}\label{sec:terminology}
    Throughout this document, we use the following terms.
    \mbi{I added two terms that were used across the doc: \transfercontract,
    \blockgenerator.}
    \begin{itemize}
        \item \contract: smart contract method activated by the user
              through a transaction.
        \mbi{With this definition, any function of an smart contract is
        called a \contract. However, in this document, the word \contract
        seems to have two different meanings: (1) a method of an smart contract
        and (2) a smart contract. I think it is best to call the functions
        of a smart contract just functions or \textbf{methods}.}
        \mbi{Following my above comment, when talking about \icc, by this
        definition, an smart contract with two different methods where one
        calls the other, would also be considered an \icc, although there is
        only one smart contract involved. Terminology seems counterintuitive to me.}
        \item \user: party initiating the \contract transaction.
        \item \gas: unit measure of computational resource.
              In this scope, \gaslimit indicates the maximum amount of
              resources allocated by a user or a contract to perform a
              computation, \gasconsumed measures incurred costs of
              computation executed, \gasunspent is the difference
              between \gaslimit and \gasconsumed.
              If a computation requires more resources than those allocated,
              then we incur in an \insufficientgas error.
        \item \gasprice: amount of \dusk that a sender is
              willing to spend per unit of gas (the amount is specified in
              Lux, where 1 Lux equals 10\textsuperscript{-9} \dusk). The higher
              the \gasprice, the more incentivized are the block generators
              to include the transaction in the next block. This way, transactions
              with high gas prices are generally confirmed more quickly.
              Because of this, \gasprice generally determines the
              transaction priority.
        \item \emphp: percentage of \gasconsumed paid to the contracts.
              The remaining percentage is paid to the \blockgenerator.
              The value of this percentage is estimated to be between 20\% and
              40\%, pending threat model analysis.
        \item \icc (inter-contract call): interaction between
              multiple contracts. ICCs cannot be called directly. As such, the
              contract methods should be tagged as \ctoc (contract-to-contract) or
              \direct. The former can only be called only by another contract,
              while the latter can only be called by a user through a
              transaction.
              %This term is also used interchangeably for those contract methods that
              %can only be called by other contract.
        \item \fee: amount of \dusk charged by a smart contract to a user, when the
              former requires payment for its services.
              The \fee is known by the transfer contract (changeable
              through a transaction), but needs to be communicated to the user,
              who approves/signs it.
              This way the possibility of fee malleability is removed, thus
              preventing a bait-and-switch attack where a contract could change
              the fee with a high priority transaction and drain the user's
              funds.
              \mbi{It is not clear to me how this attack is prevented. Is it that when
              the user calls the contract with a transaction, he includes as part of
              the transaction a signature with the fixed fee so that if the fee changes,
              then the transaction is reverted?}
        \item \contractaccount: in the scenario where the contract pays
              for gas on behalf of the user, this account is
              where fees are accrued and gas is paid.
        \item \provider: intermediary service entity between a
              \user and a \contract.
    	\item \transfercontract: the smart contract responsible for handling \dusk and
              implementing any logic related to the economics of transactions.
    	\item \blockgenerator: full-node eligible to propose a candidate block to the network.
    \end{itemize}

	\mbi{In the following section, I removed the bullets and wrote the specs in words, since
		the table already summarizes the flow in each scenario. I also made some changes to
		the table.}

    \section*{Specification}\label{sec:specifications}

	In this section, we describe how gas is handled in different scenarios. We summarize all cases in Table~\ref{tab:gas}.

	\begin{table*}[t]
		\vspace{0.3cm}
		\begin{spacing}{1.1}
			\centering
			\scalebox{0.75}{
				\begin{tabular}{| c | c | l | c | c | c |}
					\hline
					%
					\multirow{2}{*}{\textbf{Scenario}}
					& \multicolumn{3}{c|}{\textbf{Normal flow}}
					& \multicolumn{2}{c|}{\textbf{Insufficient gas}}
					\\ \cline{2-6}
					%
					& \gasunspent
					& \multicolumn{1}{c}\gasconsumed
					& \fee
					& \gasconsumed
					& \fee
					\\ \hline
					% SCENARIO 1
					\multirow{4}{*}{1. User pays gas}
					&
					& $\bullet$ (100-\emphp)\% awarded to \blockgenerator
					& \multirow{4}{*}{no fee}
					&
					& \multirow{4}{*}{no fee}
					\\
					& 100\% reverts to the \user
					& $\bullet$ \emphp\% paid to executed contracts
					&
					& 100\% to \blockgenerator
					&
					\\
					& (as change)
					& \hspace{0.2cm} distributed according to \gasconsumed
					&
					& (paid by \user)
					&
					\\
					&
					& \hspace{0.2cm} by each \icc
					&
					&
					&
					\\ \hline
					% SCENARIO 2
					&
					& $\bullet$ (100-\emphp)\% awarded to \blockgenerator
					&
					&
					& \multirow{4}{*}{returned to \user}
					\\
					2. User pays gas,
					& 100\% reverts to the \user
					& $\bullet$ \emphp\% paid to executed contracts
					& paid to
					& 100\% to \blockgenerator
					&
					\\
					contract applies fee
					& (as change)
					& \hspace{0.2cm} distributed according to \gasconsumed
					& charging contract
					& (paid by \user)
					&
					\\
					&
					& \hspace{0.2cm} by each \icc
					&
					&
					&
					\\ \hline
					% SCENARIO 3
					&
                    & $\bullet$ (100-\emphp)\% awarded to \blockgenerator
					&
					&
					& \multirow{4}{*}{returned to \user}
					\\
					3. Contract pays gas,
					& 100\% reverts to the \contract
                    & $\bullet$ \emphp\% paid to executed contracts
					& paid to
					& 100\% locked for
					&
					\\
					contract applies fee
					& (as change)
                    & \hspace{0.2cm} distributed according to \gasconsumed
					& charging contract
					& a number of epochs
					&
					\\
					&
                    & \hspace{0.2cm} by each \icc
					&
					&
					&
					\\ \hline
					% SCENARIO 4
                    &
                    & $\bullet$ (100-\emphp)\% awarded to \blockgenerator
                    &
                    &
                    & \multirow{4}{*}{returned to \user}
                    \\
                    4. Percentage of
                    & 100\% reverts to the \user
                    & $\bullet$ \emphp\% paid to executed contracts
                    & paid to
                    & 100\% to \blockgenerator
                    &
                    \\
                    Obfuscated Amount
                    & (as change)
                    & \hspace{0.2cm} distributed according to \gasconsumed
                    & charging contract
                    & (paid by \user)
                    &
                    \\
                    &
                    & \hspace{0.2cm} by each \icc
                    &
                    &
                    &
                    \\ \hline
					% SCENARIO 5
                    &
                    & $\bullet$ (100-\emphp)\% awarded to \blockgenerator
                    &
                    &
                    & \multirow{4}{*}{returned to \user}
                    \\
                    5. Autocontracts
                    & 100\% reverts to the \contract
                    & $\bullet$ \emphp\% paid to executed contracts
                    & no fee
                    & 100\% locked for
                    &
                    \\
                    & (as change)
                    & \hspace{0.2cm} distributed according to \gasconsumed
                    &
                    & a number of epochs
                    &
                    \\
                    &
                    & \hspace{0.2cm} by each \icc
                    &
                    &
                    &
                    \\ \hline
				\end{tabular}
			}
			\caption{Summary of how gas and fees are handled in each scenario.}
			\label{tab:gas}
		\end{spacing}
	\end{table*}

    \subsection*{Scenario 1: User pays gas}\label{subsec:scenario-1}
    This case mirrors the conventional gas expenditure method in most
    blockchains.
    The \user specifies a \gaslimit and a \gasprice.

    \subsubsection*{Normal flow}\label{subsubsec:scenario-1-normal-flow}

    In this case, (100-\emphp)\% \gasconsumed is transferred to the \blockgenerator,
    and \emphp\% is distributed to all touched contracts, weighted proportionally to
    the amount of \gasconsumed by each contract.

	\mbi{I have an observation for this scenario that also applies to Scenario~2. \\
	At first I thought that from the total \gasunspent, 70\% is paid back to the user
	and the remaining 30\% is split between the \direct contract and the \icc. However,
	Ed explained to me that this is not what this means, but rather the \direct contract gets
	30\% from the total \gasunspent, and then each \icc 30\% of the gas designated for it.\\
	In this case, consider this scenario: contract A gets called with 1000 gas limit,
	executes 500 of gas and calls B with 200 gas limit. Now B only spends 100 (leaving
	100 unspent). If we count it like this, then:\\
	- Block generator: 1000 - 400 = 600 \\
	- User: 70\% of 400 = 280 \\
	- Contract A: 30\% of 400 = 120 \\
	- Contract B: 30\% of 200 = 30 \\
	If the user is also the owner of contracts A and B, the user set 1000 as maximum
	gas limit, spent 600, and received 430 instead of 400. So, the user is making \dusk
	out of the blue. Is this how this works or am I missing something here? If so, I
	can't see why we want to incentivize this.
	}
    This is the default scenario, where the \user pays for gas and every
    \contract gets paid for their usage.
    The genesis contracts - the contracts handling \dusk - which are present
    from the first block - use this strategy.

    \subsubsection*{Insufficient gas}\label{subsubsec:scenario-1-insufficient-gas}
    If there is insufficient gas to execute, the \user is refunded the full amount.

    \subsubsection*{Reward to contracts}\label{subsubsec:scenario-1-insufficient-gas-reward-to-contract}
    In this scenario, contracts are rewarded directly by their \gasconsumed.
    Since the \user is the one ultimately paying, it can be considered that
    the user is paying for a contract's execution.
    This is the default behavior of most blockchains, with the added twist
    that contracts are rewarded with \gasconsumed.

    \mbi{\textit{Rewards} are not defined anywhere. Is this an extra amount of \dusk paid by
   		the user?}
	\mbi{\textit{Crossover} is a technical detail that it is also not defined and maybe adds too much
		noise to this section.}

    \subsection*{Scenario 2: User pays gas, contract applies fee}\label{subsec:scenario-2}
    This scenario mirrors the previous one, but with an additional \fee that the
    \user must add to the transaction, specified by the \contract.
    The \user sets \gaslimit, \gasprice, and the \fee paid.

    \subsubsection*{Normal flow}\label{subsubsec:scenario-2-normal-flow}
	As in Section~\ref{subsubsec:scenario-1-normal-flow}, (100-\emphp)\% of
    the \gasconsumed is transferred to the \blockgenerator, and \emphp\% is distributed
    to all touched contracts, weighed proportionally to the amount of \gasconsumed.

    The \fee, on the other hand, is paid directly to the \contract by the \user.

	\mbi{I have the same observation as before.}

    \subsubsection*{Insufficient gas}\label{subsubsec:scenario-2-insufficient-gas}
    In this case, 100\% of the \gas is returned to the \user, along with the
    \fee.

    \subsection*{Scenario 3: Contract pays gas, contract applies fee}\label{subsec:scenario-3}
    This scenario is similar to the previous one\ref{subsec:scenario-2}, in that
    it allows a contract to specify a \fee that the \user must pay.
    It differs however in that the \contract pays for the gas consumed during a call,
    as opposed to the previous scenario where the \user pays for the gas.
    This scenario allows a contract set a fixed \fee paid by the \user.

    This allows contracts to effectively subsidize a \user's gas costs, and may
    be used by the contract to incentivize users to use their services.
    Contract's using this scenario may wish to set a \fee higher than the gas
    costs themselves, such that the contract earns a profit from the transaction.
    The situation where this is not possible is described below in the
    \textit{User pays fees lesser than gas paid by contract} section below.

    \subsubsection*{Normal flow}\label{subsubsec:scenario-3-normal-flow}
    The accounting of \gas is similar to the previous scenario, with the exception
    that the \gasconsumed being paid by the \contract as opposed to the \user.
    (100-\emphp)\% of the \gasconsumed is transferred to the \blockgenerator, and
    \emphp\% is distributed to all touched contracts, weighed proportionally to the
    amount of \gasconsumed.

    \subsubsection*{Insufficient gas}\label{subsubsec:scenario-3-insufficient-gas}
    In the same way as the other scenarios, if there is \insufficientgas for the
    execution to complete, the \user is refunded the full amount of the \fee.
    However, in this case, since the \contract is paying for the gas, the \contract
    receives is refunded the amount that it would have paid for the gas.

    \subsubsection*{Setting gas price and limit}\label{subsubsec:scenario-3-setting-gas-price-and-limit}
    In the previous scenarios the \user sets the \gasprice and \gaslimit, and
    sends a transaction to the network without any interaction with the \contract.
    Given that, in this scenario, the \contract is paying for the gas, it becomes
    necessary for the \contract to be able to set the \gasprice and \gaslimit it
    is willing to pay for a specific transaction.

    This imposes that the contract must be able to inform the \blockgenerator of
    the \gasprice and \gaslimit it is willing to pay for a given transaction,
    meaning that the binary interface of the \contract with the \blockgenerator
    must be extended to support this functionality.
    For any of the previous calculations to be possible, the protocol must make
    data, such as low, average, and high gas price, available to the \contract.
    This data will be served, like any other data, using a host function - a
    function that is not part of the \contract, but is made available to it
    by the node.

    \subsubsection*{User pays fees lesser than gas paid by contract}\label{subsubsec:scenario-3-user-pays-fees-lesser}
    It is possible for a \user to pay a \fee that is less than the \gasconsumed.
    In this case, the \contract paying for gas and levying a \fee will be
    operating at a loss.
    Such a situation may be exactly what the \contract intends, however, to
    ensure that no abuse is possible, the \contract must be able to specify
    whether it is willing to operate at such a loss.
    If the \contract is not willing to operate at a loss, and the \user pays
    a \fee that is less than the \gasconsumed, the transaction is treated in
    the same way as if it ran out of gas\ref{subsubsec:scenario-3-insufficient-gas}.

    \subsubsection*{Service provider contract}\label{subsubsec:scenario-3-service-provider-contract}
    This scenario is particularly useful for so-called service \provider contracts.
    These are contracts designed to provide a service to \users, such as
    generating a zero-knowledge proof, a subscription service, off-chain
    payments, etc.
    In particular, a zero-knowlege proof \provider offers several advantages:

    \begin{enumerate}
        \item \textbf{Reduced note proliferation}: fees managed by a contract
        can be accumulated within a single note, which can be withdrawn
        later.
        \item \textbf{Instant notification of new services}: wallets are
        immediately notified of new services and their fees through a
        simple event broadcast by registrars.
        \item \textbf{Support for off-chain services}: the contract allows
        notifications and incentives for off-chain services, such as
        community-run proof sequencers and provers.
    \end{enumerate}

    \subsection*{Scenario 4: Contract charges a percentage of an obfuscated amount}\label{subsec:scenario-4}
    This scenario is similar to the second scenario\ref{subsec:scenario-2}, in
    that the contract charges a \fee to the \user.
    The difference lies in the fact that in this scenario the \fee is a
    percentage of an obfuscated amount.
    Transactions of \dusk can be obfuscated, meaning that the amount being
    transacted is hidden from the network.
    In this scenario the \contract accepts a percentage of that amount, and
    is assured of its validity using a zero-knowledge proof.

    \subsection*{Scenario 5: Autocontracts}\label{subsec:scenario-5}

    Leveraging the economic model, defined by the previously described scenarios,
    we are in a position to introduce a new type of contract: autocontracts.
    Autocontracts are contracts that are executed automatically when a specific
    event occurs, leveraging the economic model to pay for their own gas, using
    Scenario 3\ref{subsec:scenario-3}.

    The main benefit of smart contracts is that they can be leveraged to implement
    complex logic that can automatically executed.
    This is particularly useful in the context of decentralized finance, where
    functionality such as limit orders, stop-loss orders, etc.\ can be implemented.

    In its essence, reactive applications can now be implemented on-chain, without
    a need for a centralized service to monitor the state of the blockchain and
    react to events.

    \subsubsection*{Considerations}\label{subsubsec:considerations}

    \textbf{Execution Priority}: It is possible to envision autocontracts being
    executed in a few different orders.
    The first could be a simple ``contract ID order'', where the contracts are
    executed in order of their contract ID.\
    This would be the simplest approach, but would not allow for any contract
    to be prioritized over another.
    The second approach could be a ``contract fee order'', where the contracts
    are executed in the order of which pays the most fees.
    This approach would be is more complex, but it would allow for contracts
    to be prioritized over others.

    Any approach taken would need to be carefully considered, as it would
    become a part of the protocol.\\

    \textbf{Moment of Execution}: There are multiple possible moments in which an
    autocontract could be executed.
    The first possibility would be immediately after the triggering event.
    This would be the simplest approach, but it could lead to some significant
    drawbacks, such as adding to the block gas limit, and potentially preventing the
    execution of other transactions.
    The second approach would be to execute the autocontracts at the end of the
    block, after all other transactions have been executed.
    This would be a more complex approach, but might allow the \blockgenerator to
    game the system by including transactions that would cause the autocontract to
    execute in a way that would be beneficial to the \blockgenerator.

    Any approach taken would need to be carefully considered, as it would
    become a part of the protocol.

    \section*{Model Discussion}\label{sec:discussion}
    Dusk differs from traditional blockchain models by allowing gas costs to be
    paid by contracts rather than users.
    This approach has several advantages, including:

    \begin{itemize}
        \item It fundamentally improves user experience.
        \item It solidifies long-term developer commitment through a sustainable
              revenue stream.
        \item It departs from conventional models where network congestion sets
              the price, opting instead for a cost-effective utilization-based
              approach.
        \item It shifts the focus from speculative tokens to genuine service
              utility, minimizing scams, and conferring special advantages to
              financial institutions by aligning with regulatory compliance
              requirements.
    \end{itemize}

    Adopting the economic protocol at Dusk's base layer rather than at the
    application level yields unique strategic benefits.
    It promotes a unified approach to the UX of wallets and clients, and
    incentivizes novel feature creation through a standardized base layer.
    In fact, the very concept of autocontracts, introduced in Scenario 5~\ref{subsec:scenario-5}
    emerges from this setup and would not be possible at application level,
    offering a truly unique mechanism to create a scalable market for optimized
    smart contracts.

    Dusk's economic model stands out from traditional approaches by allowing
    users to pay smart contracts directly and specify their preferred payment
    method.
    This model permits the transfer of gas costs to contracts rather than users,
    aligning more closely with conventional scenarios where service providers
    bear infrastructural costs.

    Albeit other blockchains are trying to create incentives for smart contract
    developers (owners), they all tend to keep the current gas philosophy quite
    unchanged, that is users pay for gas, i.e.\ part of gas spent by the user is
    thus paid to contracts' owner.\\

	\scalebox{0.87}{
    \begin{tabular}{|m{0.3\linewidth}|m{0.7\linewidth}|}
        \hline
        \textbf{Configuration} & \textbf{Contract owner incentives} \\
        \hline
        Traditional & No incentive (owner's margin is \textbf{absent}, or
        depends on the smart-contract's specific logic). \\
        \hline
        Gas-subsidized contracts & Owner is subsidized by users' gas: a
        percentile of the gas fees no longer goes to the \blockgenerator, but
        to the invoked contracts. \\
        \hline
        Gas-and-mint subsidized contracts & Owner is subsidized by users' gas
        and also part of the block reward. \\
        \hline
    \end{tabular}}

    \subsection*{Prior Art}\label{subsec:prior-art}
    Some blockchains are also trying to create incentives for smart contract
    developers/owners.
    Fantom\footnote{https://docs.fantom.foundation/funding/gas-monetization} and NEAR\footnote{https://docs.near.org/concepts/basics/transactions/gas} propose
    similar approaches to the one presented in this document, albeit with
    slightly different implementations.

\end{document}
